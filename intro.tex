\section{Introduction}

Android and mobile devices and smartphones are becoming increasingly popular.  The number of mobile phones
sold has surpassed the number of laptops, and in 2014 it was XXX. Google is reported to have more than
a billion active users of Android-based devices.  As their popularity increases, so does their value as
a target for malware.  There are many possible risks.
Many people use smartphones for financial transactions.  Attackers could get personal
information for systems administrators and CIOs from their phones and use that for spear phishing.  
%mention security and privacy issues 

The more sophisticated mobile devices become, the more complex the threat model is, and the more opportunities
for vulnerabilities.  Finding viruses and other malware using software signatures is less and less likely to
work.  What is needed is a system-wide  aproach that involves assessing trust for the different
components by detecting anomalies in sensor behavior and history of the component.  
In recent research done by Hoffman, et al.~\cite{hoffman}, the authors developed a hierarchical 
mechanism that is scalable and expandable for evaluating security of
Android smartphones by investigating various sources of information regarding 
the mobile devices. 
%
The three sources of information that they integrate are:
analysis of installed applications using metadata provided by the Google Play store,
usage information coming from security tools embedded in the OS, and 
validation of the device by inspection of sensor data.  
The sensors in this context refer to any function that
measures the physical state of the smartphone.  This can include geolocation, accelerometers, CPU 
utilization, and battery charge.

In this paper we focus on the second and third sources: inspection and comparison of sensor values.
\yanyan{need more text..}

The contributions of this paper include. \yanyan{complete..}