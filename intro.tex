\section{Introduction}

Android-based mobile devices and smartphones are becoming increasingly popular.  The number of mobile phones
sold has surpassed the number of laptops, reaching XXXin 2014~\cite{}.  Google is reported\cite{} to have more than
a billion active users of Android-based devices.  As their popularity increases, so does their value as
a target for malware injection.  
This is particularly true for low cost end smartphones sold in developing countries. According to \cite{}
some vendors there intentionally create conditions facilitating various security violations in these devices.
\reznik{I believe I know the ref but I need some time to find it out}
 There are many possible risks, associated with using compromised devices.  Nowadays due to universal interconnection and 
interdependence the 
possible compromise of a mobile device will affect not only applications executed on it and its users but all other parts 
of a networked computer and communication infrastructure.
\eat {There are many possible risks.
Many people use smartphones for financial transactions. 
 Attackers could get personal information for systems administrators and CIOs from their phones and 
 use that for spear phishing.  mention security and privacy issues}  % end comment
With the development of the mobile communication platforms, which share different devices resources, applications and data, 
this trend will become stronger and stronger. 
In order to facilitate an introduction of the common cyberinfrastructure and promote collaboration between various devices 
and users, we need to develop models and methods that will evaluate trust, which other infrastructure parts may have in a 
particular device or an application, and present this evaluation to other participants. The trust assessment values could 
be applied to optimize data collection and communication schemes in order to satisfy multiple criteria such as overall 
system performance and/or resource consumption. Also, the user of a device may benefit from the trust evaluation as it might 
provide useful information about areas in need of improvement, and it could combined with other techniques for
 non-signature based intrusion detection.

The more sophisticated mobile devices become, the more complex the threat model is, and the more opportunities
for vulnerabilities.  Finding viruses and other malware using software signatures is less and less likely to
work.  What is needed is a system-wide  approach that involves assessing trust for the different
components by detecting anomalies in sensor behavior and history of the component.  
\eat{
In recent research done by Hoffman, the authors developed a hierarchical 
mechanism that is scalable and expandable for evaluating security of
Android smartphones by investigating various sources of information regarding 
the mobile devices. 
The three sources of information that they integrate are:
analysis of installed applications using metadata provided by the Google Play store,
usage information coming from security tools embedded in the OS, and 
validation of the device by inspection of sensor data.  
The sensors in this context refer to any function that
measures the physical state of the smartphone.  This can include geolocation, accelerometers, CPU 
utilization, and battery charge.} % end comment

\weiss{Leon, are the following paragraphs what you had in mind?}
\eat{
McKnight and Chervany, What is Trust? A Conceptual Analysis and an Interdisciplinary Model

Paul England, Butler Lampson, John Manferdelli, Marcus Peinado, Bryan Willman: A
Trusted Open Platform. IEEE Computer Scciety, p55-62, July 2003.

Adrian Baldwin, Simon Shiu: Hardware Security Appliances for Trust. In Proceedings of
the First International Conference of Trust Management (iTrust 2003), Crete, Greece, May
2003.

Daniel W. Manchala: Xerox Research and Technology. E-Commerce Trust Metrics and
Models. IEEE Internet Computing, vol.4, no.2 p.36-44 (2000).

Mui Lik, Mohtashemi Mojdeh, Halberstadt Ari: A Computational Model of Trust and Repu-
tation. In Proc. Of the 35th Annual Hawaii International Conference on System sciences, 7-
10 (Jan. 2002), Big Island, HI, USA.

 Mogens Nielsen,  Karl Krukow A Bayesian Model for Event-based Trust, 2007

Zheng Yan and Piotr Cofta, “A Mechanism for Trust
Sustainability among Trusted Computing Platforms", In Proceedings of the 1st
International Conference on Trust and Privacy in Digital Business
(TrustBus2004), LNCS Vol. 3184/2004, pp. 11-19, Spain, September 2004.
} % end comment

The concepts of trust, trusted code base and trusted computing platform (TCP) have been discussed by many others~\cite{}.
however, it seems that the problem  of quantification is largely unsolved, especially with respect to complex systems.
If one looks at individual platform within a system, one can simplify the problem by establishing a trusted code base (TCB).
This is an important principle of secure design, in which the size of the TCB should be as small as possible, yet contain
sufficient functionality that all other software components on the platform can be built on top of it.  Security restrictions on 
these other higher levels are imposed using reference monitors.  \weiss{How does the Android OS provide security checks?}

There are two problems that must be solved for multi-platform systems: (1) combining trust metrics from the individual 
TCPs and (2) understanding and measuring trust over time.
Yan and Cofta~\cite{yan_2004} point out that even if one can establish a level of trust between two platforms, 
the more complex problem is sustaining trust over time.  This paper focuses primarily on the first problem of combining
information to produce a trust metric for the whole system.  Yet, the nature of our work points to ways in which 
this could be used over time.  By comparing sensor data from different systems, one can periodically check for consistency.

This paper presents the development of the novel hierarchical framework that enables the  evaluation of trust for a network of 
mobile devices. As a complex concept, the trust depends on numerous factors. The hierarchical structure allows for the
inclusion of various trust evaluation systems used to assess diverse trusted components as well as their integration in 
order to produce the cumulative trust score. Also, it facilitates the framework extension by inclusion of new trust 
component evaluation, facilitating evaluation results improvement for a particular device as well as the evaluation 
adaptation to diverse devices and applications. \weiss{I don't understand the previous sentence}
The paper describes the version developed for using on Android-based devices.

The framework design principles and an overall architecture are described in section II. Also, this section 
 briefly describes a few of the trust evaluation metrics included in the current implementation,
while more details about a few of them are provided in sections III-V. In particular, section III discusses metrics whose 
calculation is based on measuring the device utilization such as voltage supply, CPU and network bandwidth usage. 
Section IV describes metrics based on multiple sensors that impact the level of privacy supported. 
Section V discusses the possible ways in which trust evaluation can be verified and adjusted, and it 
describes some practical cases that have been implemented.
