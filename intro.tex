\section{Introduction}

Android-based mobile devices and smartphones are becoming increasingly popular.  The number of mobile phones
sold has surpassed the number of laptops, reaching 1.3 billion in 
2014~\cite{market}.  Google is 
reported to have more than a billion active users of 
Android-based devices~\cite{android-users}.  
As their popularity increases, so does their value as
a target for malware injection.  
This is particularly true for low cost smartphones sold in developing countries. According to \cite{zheng2014droidray}
some vendors there intentionally create conditions facilitating various security violations in these devices.
 There are many possible risks, associated with using compromised devices.  Nowadays due to universal interconnectivity and 
interdependence the 
possible compromise of a mobile device will affect not only applications executed on it and its users but all other parts 
of a networked computer and communication infrastructure.
\eat {There are many possible risks.
Many people use smartphones for financial transactions. 
 Attackers could get personal information for systems administrators and CIOs from their phones and 
 use that for spear phishing.  mention security and privacy issues}  % end comment
With the development of the mobile communication platforms, which share different devices resources, applications and data, 
this trend will become stronger and stronger. 

In order for mobile devices such as smartphones to communicate with each other, download apps and preserve security, they need to
be able to compute trust metrics.   Trust can be modeled at 
multiple levels, e.g. an application, a device (hardware and software), or a network of devices.  Ultimately, we need to 
integrate these into a single conceptual framework in order to use them, but we can also use them to verify trust metrics
for individual components.
The trust evaluation  could 
be applied to optimize data collection and communication schemes in order to satisfy multiple criteria such as overall 
system performance and/or resource consumption, subject to constraints based on security and privacy requirements.
Also, the user of a device may benefit from the trust evaluation as it might 
provide useful information about areas in need of improvement, and it could be combined with other techniques for
 non-signature based intrusion detection.

The more sophisticated mobile devices become, the more complex the threat model is, and the more opportunities there are
for vulnerabilities to appear.  Trust evaluation should be sensitive to the detection of viruses and other malicious
agents in a system.
However, finding viruses and other malware using software signatures is less and less likely to
work.  Signature based intrusion detection systems have to be complemented with a system-wide  approach that 
involves assessing trust for the different
components by detecting anomalies in sensor-originated data.
\eat{
In recent research done by Hoffman, the authors developed a hierarchical 
mechanism that is scalable and expandable for evaluating security of
Android smartphones by investigating various sources of information regarding 
the mobile devices. 
The three sources of information that they integrate are:
analysis of installed applications using metadata provided by the Google Play store,
usage information coming from security tools embedded in the OS, and 
validation of the device by inspection of sensor data.  
The sensors in this context refer to any function that
measures the physical state of the smartphone.  This can include geolocation, accelerometers, CPU 
utilization, and battery charge.} % end comment


\eat{
- McKnight and Chervany, What is Trust? A Conceptual Analysis and an Interdisciplinary Model
- Paul England, Butler Lampson, John Manferdelli, Marcus Peinado, Bryan Willman: A
Trusted Open Platform. IEEE Computer Scciety, p55-62, July 2003.
- Adrian Baldwin, Simon Shiu: Hardware Security Appliances for Trust. In Proceedings of
the First International Conference of Trust Management (iTrust 2003), Crete, Greece, May
2003.
- Daniel W. Manchala: Xerox Research and Technology. E-Commerce Trust Metrics and
Models. IEEE Internet Computing, vol.4, no.2 p.36-44 (2000).
- Mui Lik, Mohtashemi Mojdeh, Halberstadt Ari: A Computational Model of Trust and Repu-
tation. In Proc. Of the 35th Annual Hawaii International Conference on System sciences, 7-
10 (Jan. 2002), Big Island, HI, USA.
- Mogens Nielsen,  Karl Krukow A Bayesian Model for Event-based Trust, 2007
- Zheng Yan and Piotr Cofta, “A Mechanism for Trust
Sustainability among Trusted Computing Platforms", In Proceedings of the 1st
International Conference on Trust and Privacy in Digital Business
(TrustBus2004), LNCS Vol. 3184/2004, pp. 11-19, Spain, September 2004.
} % end comment

The concepts of trust and trust evaluation have been discussed by many others~\cite{jing2014riskmon,shabtai2010google,zheng2014droidray}; 
however, it seems that the problem  of quantification is largely unsolved, especially with respect to complex systems.
 Yet, the nature of our work points to ways in which 
this could be used to produce secure or trustworthy systems.  
This paper presents the development of the novel hierarchical model that enables the  evaluation of trust for a network of 
mobile devices. Trust evaluation depends on numerous factors. The hierarchical structure allows for the
inclusion of various trust evaluation systems used to assess diverse trusted components as well as their integration in 
order to produce a cumulative trust score. Also, it allows for extending the framework by inclusion of new trust metrics
and facilitates both self-evaluation for a particular device as well as the collaborative evaluation of diverse devices 
and applications. 
The paper describes the version developed for using on Android-based devices.

Section II describes the framework design principles and an overall architecture. Also, this section 
 briefly describes a few of the trust evaluation metrics developed for individual apps and
smartphones that are included in the current implementation,
while more details about a few of them are provided in sections III-V. In particular, section III discusses metrics 
for individual apps based on measuring resource utilization such as voltage supply, CPU and network bandwidth usage,
which can be done on the smartphone.
Section IV describes metrics based on multiple sensors that impact the level of privacy supported, and how 
privacy-enhancing tools can interact with trust evaluation.
Section V discusses the possible ways in which trust evaluation can be verified and adjusted, based on multiple sources
of data.  In particular, we show data collected simultaneously from a smartphone and the speedometer of an 
automobile.
