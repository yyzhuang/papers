\section{Introduction}

Android-based mobile devices and smartphones are becoming increasingly popular.  The number of mobile phones
sold has surpassed the number of laptops, reaching XXXin 2014~\cite{}.  Google is reported\cite{} to have more than
a billion active users of Android-based devices.  As their popularity increases, so does their value as
a target for malware injection.  
This is particularly true for low cost end smartphones sold in developing countries. According to \cite{}
some vendors there intentionally create conditions facilitating various security violations in these devices.
\reznik{I believe I know the ref but I need some time to find it out}
 There are many possible risks, associated with using compromised devices.  Nowadays due to universal interconnection and 
interdependence the 
possible compromise of a mobile device will affect not only applications executed on it and its users but all other parts 
of a networked computer and communication infrastructure.
\eat {There are many possible risks.
Many people use smartphones for financial transactions. 
 Attackers could get personal information for systems administrators and CIOs from their phones and 
 use that for spear phishing.  mention security and privacy issues}  % end comment
With the development of the mobile communication platforms, which share different devices resources, applications and data, 
this trend will become stronger and stronger. \weiss{should we mention Sensibility?}
In order to facilitate an introduction of the common cyberinfrastructure and promote collaboration between various devices 
and users, we need to develop models and methods that will evaluate trust, which other infrastructure parts may have in a 
particular device or an application, and present this evaluation to other participants. The trust assessment values could 
be applied to optimize data collection and communication schemes in order to satisfy multiple criteria such as overall 
system performance and/or resource consumption. Also, a device user may benefit from the trust evaluation as it might 
provide useful information about areas in need of improvement. It could applied in novel non-signature based intrusion 
detection techniques too.

The more sophisticated mobile devices become, the more complex the threat model is, and the more opportunities
for vulnerabilities.  Finding viruses and other malware using software signatures is less and less likely to
work.  What is needed is a system-wide  approach that involves assessing trust for the different
components by detecting anomalies in sensor behavior and history of the component.  
In recent research done by Hoffman, \weiss{not sure what to say here.} the authors developed a hierarchical 
mechanism that is scalable and expandable for evaluating security of
Android smartphones by investigating various sources of information regarding 
the mobile devices. 
%
The three sources of information that they integrate are:
analysis of installed applications using metadata provided by the Google Play store,
usage information coming from security tools embedded in the OS, and 
validation of the device by inspection of sensor data.  
The sensors in this context refer to any function that
measures the physical state of the smartphone.  This can include geolocation, accelerometers, CPU 
utilization, and battery charge.

\reznik{Here we need to write one-two pars bib review about trust models and evaluation metrics.}

This paper presents the development of the novel hierarchical framework that allows the trust evaluation of the network of 
mobile devices. As a complex concept, the trust depends on numerous factors. The hierarchical structure allows for an 
inclusion of various trust evaluation systems used to assess diverse trust components as well as their integration in 
order to produce the cumulative trust score. Also, it facilitates the framework extension by inclusion of new trust 
component evaluation, facilitating evaluation results improvement for a particular device as well as the evaluation adaptation to diverse devices and applications. The paper describes the version developed for using on Android-based devices.

The contributions of this paper include: \yanyan{complete..}
the integration of sensor measurments from multiple devices, ...
