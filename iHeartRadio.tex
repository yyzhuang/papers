\section{Measuring app behavior}

Anomalous behavior of a smartphone may be due to malicious adulteration of an application or
the operating system, or installed during the hardware manufacturing chain.
In order to be able to detect this
behavior, we first need to establish baseline measurements for 
normal behavior. We focus on metrics that are observable 
without requiring inspection of the app's source or binary code
because obfuscation techniques are very sophisticated and will continue to be developed.
We have enhanced existing debugging with additional internal sensor measurements to inspect app behavior.
Aggregate metrics, such as battery voltage and CPU usage are easy to collect.  However,
they can also be influenced by random and extraneous factors.

\subsection{Battery voltage}
A proxy metric for the amount of remaining 
energy in the battery. Any kind of activity on the device will result 
in a change of battery voltage, but the resolution of readings (both 
in time and in voltage) only allows for coarse, averaged measurements 
under general, non-lab conditions.

If the circumstances can be controlled tightly, then approaches like 
\textit{Eprof} \cite{pathak2012fine} can be used to estimate energy consumption 
of individual activities and assign credit to likely originators.


\begin{table}[ht]
\centering 
\scriptsize
\begin{tabular}{|l||l|l|l|}
    \hline
    {\bf Time} & {\bf 15s} & {\bf 60s} & {\bf 10min} \\
    \hline
    \% of total battery use with ads (avg)  & 81.4\%   & 53.0\%   & \\
    \hline
    \% of total battery use with ads (max) & 86.0\%  & 61.5\%  & 11.9\%  \\
    \hline

\end{tabular}
\caption{A comparison of battery use with anomalous adware compared with normal.
The ad was 15 seconds.}
\end{table}

\subsection{CPU usage}

\begin{figure*}[ht]
\centering
\includegraphics[width=6in]{no_AdsDisplay_10s.png}
\caption{A typical session in Traceview, the execution log viewer.
WebViewCoreThread is busy, but main thread's workload is very light.  main is reponsible for 
the user interface, in general.  WebView is reponsible for rendering ads using the webkit library}
\label{fig:cpu-no-ad}
\end{figure*}

\begin{figure*}[ht]
\centering
\includegraphics[width=6in]{traceview.png}
\caption{A session in Traceview, with anomalous Ads.  Notice that main is doing much more work and
there are other threads that are getting significant CPU time.}
\label{fig:cpu-ad}
\end{figure*}

Similar to battery voltage, monitoring the overall CPU usage 
(which is an operation requiring no special privileges on Android) 
can be used to get a coarse-grained overview of consumption. 
However, the existing debugging and tracing interfaces permit 
finer-grained views as well. Assuming an app is not designed to 
evade or complicate debugging deliberately, the 
\textit{Dalvik Debug Monitoring Service} (DDMS) 
% https://developer.android.com/tools/debugging/ddms.html
provides syscall-level insight.

Figure~\ref{fig:cpu} shows a typical  
session in Traceview, the execution log viewer.
\yanyan{did you mean Figure~\ref{fig:cpu-no-ad}? how about 
Figure~\ref{fig:cpu-ad}? We should mention both.}
% Current figure (MD5 46fbce842e071dfb8f321c9645c6b411) courtesy of Tao Li.
The app under scrutiny comprises several threads with different 
temporal activity patterns. Some threads bear names indicative 
of their performed functions. Colored bars in the threads' activity 
timelines further detail call-level interactions with app and system 
libraries, with the height of sub-bars proportional to the frequency 
of specific calls.

Other log views (not shown) list all of an app's threads, show the 
call stack for each, and display cumulated and individual CPU time 
consumption, and relative usage.


%%% http://www.sigmobile.org/mobisys/2012/program.php#ses5

\subsection{Network usage}

Android provides a number of built-in features allowing the observation 
of a device's network conditions for any app granted the 
\texttt{ACCESS\_NETWORK\_STATE} permission. 
The \texttt{ConnectivityManager} class lets an app discover the 
current connectivity status and type (WiFi, 3G, Bluetooth, Ethernet). 
For cellular access such as LTE or 3G, the \texttt{TelephonyManager} 
class makes available further detail. The stateful nature of cellular 
data connectivity is reflected by various indicators of data activity, 
thereby allowing any app to detect when other apps transfer data over 
the cellular interface.  The Application Resource Optimizer project
(ARO\footnote{See \url{https://github.com/attdevsupport/ARO}}) and \cite{Ricciato2010551} 
provide further insights into what is 
essentially radio resource control (RRC-based type of diagnostics).

%%% https://developer.android.com/reference/android/telephony/TelephonyManager.html (See NETWORK_TYPE_* and DATA_* states)
%%% https://developer.android.com/reference/android/net/ConnectivityManager.html (See TYPE_*)
%%% https://developer.android.com/reference/android/net/NetworkInfo.html#isConnected%28%29

If the device is \textit{rooted} (i.e., system-level administrator 
privileges are available to the user launching an app), standard 
packet tracing tools such as \texttt{tcpdump} can be used to 
record the exact data transferred across network interfaces. 
However, the precondition is not met on almost any commercial stock 
firmware.
% Note: tcpdump doesn't allow inspection of encrypted content either. 
% There are approaches adding a system-level certificate to MITM on 
% HTTPS connections see for instance https://github.com/egirault/googleplay-api ,
% but I think this also requires root.

When packet-level tracing on the device is infeasible, the network 
connection of the device might be tapped instead. A natural place 
for this would be a WiFi router acting as the device's gateway. 
Neither on-device nor on-path packet tracing allow decryption of 
HTTPS and other encrypted network traffic. However, at least for 
HTTPS implementations using the system libraries, deliberate 
man-in-the-middle (MITM) attacks on traffic may be performed 
by adding a self-provided certificate to the system's certificate 
storage, and redirecting outgoing HTTPS traffic to a local proxy 
server using that certificate.
