\section{Conclusion}
With the growing number of mobile devices owned by ordinary citizens all around the world and an increase 
in network connectivity and data provided, the problem of evaluating trust becomes 
more complex.  The risks to
the quality and security of data and devices become greater and more important to investigate and evaluate. The 
umbrella trust evaluation 
framework that we have developed includes procedures and tools to evaluate trust in mobile devices and applications, in particular in Android based smartphones. 
The  hierarchical structure of the framework allows for incorporating multiple diverse factors affecting trust evaluation as well as facilitating its extension.
The current framework version includes evaluation procedures based on the analysis of the following factors: the applications installed and executed, the device’s operating system settings and configuration, the level of privacy supported by the devices and applications, the patterns of the battery drain and CPU and network bandwidth usage,
and the corroboration of sensor data by external sensors.

This empirical study demonstrates a significant difference between the change of voltage in the cases of an execution 
of normal applications and the same applications with embedded advertisements.  Additional observatoins produced distinctive patterns of the CPU and network use.
The framework includes not only the trust metrics but also the rules for their combination into a trust
evaluation and the methods of its verification. 
Trust verification and adjustment could be performed through comparison of data received from different data sources on the 
same device as well as by comparing the data originated from various devices. A significant discrepancy between data 
originated from various sources should results in decreasing trust assessment for the corresponding data sources and devices. 
In the paper, we have described a comparative study of the speed measurements obtained from a car speedometer sensor 
and the speed calculated 
with the GPS location sensor measurements. This illustrates how the framework could be employed not only for trust evaluation 
but also for detecting various anomalies, which might reflect malicious attacks against the mobile devices.
