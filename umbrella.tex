\section{FRAMEWORK UMBRELLA ARCHITECTURE AND DESIGN}
%This is a summary of section 2 in the Hoffman paper

A hierarchical trust analysis can look at the entire system and combine measurements from multiple sources, making
it more powerful than measuring a single component or layer.
In this section we discuss a comprehensive mechanism that provides a scalable and extendable methodology of security evaluation and analysis for Android based mobile devices. 

Quality and security evaluation is a complex subject, which depends on multiple characteristics, e.g. sensor accuracy, the rate of encrypted messages, and/or the probability of a system's breakdown over a given period of time. Its evaluation should integrate various metrics ranging from the accuracy and reliability of the data sources to the security of the procedures and tools used. The major research challenge of the framework design is integrating the numerous metrics needed to characterize a device and its security while working with limited resources and processing power. We address this challenge by hierarchically structuring the security metrics composition as well as by designing a specialized calculus to evaluate the overall security. 

Therefore, the major innovative emphasis in our framework design is put on the integration of a wide variety of indicators and their evaluation procedures. The framework procedures output the overall data quality evaluation indicators and additionally calculate the individual product of metrics characterizing system features which are then used to produce recommendations for improvement. The application will facilitate decision making, improve performance and increase accountability through the collection, analysis, and reporting of relevant performance-related data. This design facilitates the framework extension and makes it accessible for other metrics inclusion as well as an easy modification and improvement.
Current implementation of this framework provides the following metric functionality: 
\yanyan{according to intro, we only look at 2 and 3. Better merge this sec with intro.}

\begin{enumerate}
\item Analysis of the installed applications through the application specific metadata provided by the Play store. Applications represent the largest security and privacy risk to a device and user's data. The data provided by the Play store leverages the experiences of millions of users and holds all data associated with the distribution of an application including its associated documentation. The Play store also provides meta-information about applications which provides useful characteristic data about an application. This data can be used to assess an individual applications risk. Rules were generated to classify each application into a risk impact class based on this meta-data. The combined security classes of all applications installed on a device would be used to create a security risk rating for the whole device.

\item The usage verification of security tools embedded into the operating system and proper preventative security practices. Android provides users with many different tools which increase the security and privacy of their devices in addition to updates to patch exposed vulnerabilities. When properly used, these tools improve the security of the devices. In order to gather a comprehensive overview of the software running on a mobile device an analysis of the operating system and user settings is performed. First the operating system is checked to confirm that it is running the most recent version available. Second the personal security settings on the device are examined to determine if the user is utilizing the appropriate tools to secure the device. These operating system verification checks combined generate a score, which is used in the security and privacy framework operation result.

\item Validation of the device evaluation through inspection of the sensor data. Detection of the compromised devices can sometimes be determined by the analysis of the spurious output of its internal sensors. Verifying the validity of the sensors can detect security and quality problems of the device that would be missed by the other subsections. Most mobile devices now come equipped with a variety of sophisticated sensors which are capable of very accurate measurements of their surrounding environment. As the data from these sensors are used in more security critical applications the importance that these data remain accurate and legitimate should not be underestimated. For example, data from the GPS sensor can be verified to be trustworthy and assigned a security rating. The combination of ratings from all sensors would produce the devices sensor security risk score.
\end{enumerate}


Unlike other reported tools available, our framework has an umbrella structure that allows for an integration of various diverse security evaluation mechanisms and results. This open architecture provides an opportunity for an easy and simple extension and inclusion of various tools as well as a combination or diverse target areas. At the same time, the self-learning ability allows for its optimization towards a particular device and a criteria set. Each of these procedures given above generates a security risk rating, which is then integrated together as part of the umbrella framework. This framework takes into account the varied landscape of mobile devices and is designed to be flexible and easily adaptable to the changing security environment. Based on this design and contribution of each of the procedures could be adjusted depending on the target.
\yanyan{I don't think we do self-learning or risk rating. Delete? Otherwise IMHO would be considered 
as over claim by reviewers.}